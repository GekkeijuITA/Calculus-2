\documentclass[12pt, a4paper]{report}
\usepackage[utf8]{inputenc}
\newcommand\preamble{
    \usepackage[italian]{babel}
    \usepackage{geometry}
    \usepackage{amsmath}
    \usepackage{amssymb}
    \usepackage{graphicx}
    \usepackage{ulem}
    \geometry{margin=2cm}
    \usepackage{listings}
    \usepackage{xparse}
    \usepackage{expl3}
    \usepackage{tikz}
    \usetikzlibrary{calc}
    \let\olditemize\itemize
    \renewcommand\itemize{\olditemize\setlength\itemsep{0em}}
}
\newcommand{\tikzmark}[1]{\tikz[baseline,remember picture] \coordinate (#1) {};}
\newcommand{\customfbox}[1]{
    \begin{center}
        \noindent\fbox{\parbox{\dimexpr\linewidth-2\fboxsep-2\fboxrule\relax}{\centering #1}}
    \end{center}
    }
\input{titlePage.tex}
\preamble

\begin{document}
\customTitlePage{Calculus 2}{Lorenzo Vaccarecci}{Anno Accademico 2024/2025}{Università degli Studi di Genova}
\newpage
\tableofcontents
\chapter{Formula di Taylor}
    \customfbox{
        $f:I\subseteq \mathbb{R}\rightarrow \mathbb{R}$ 
        \\ derivabile infinite volte su $I$
        }
    Polinomio di Taylor di $f$ centrato in $x_{0}$ di ordine $n$:
    \begin{equation*}
        \taylor{f}{x_{0}}{n}(x) = f(x_{0})+f'(x_{0})(x-x_{0})+f''(x_{0})\frac{(x-x_{0})^{2}}{2!}+\ldots+f^{(n)}(x_{0})\frac{(x-x_{0})^{n}}{n!} 
    \end{equation*}
    $\taylor{f}{x_{0}}{n}(x)$ è un polinomio nella variabile $x$ di grado $\leq n$.\\
    La formula di Taylor ha un duplice scopo:
    \begin{itemize}
        \item Permette di approssimare $f$ con un polinomio
        \item Permette di approssimare il valore di $f$ in un punto $b\in I$ con il valore di $f$ in $x$ che sta vicino a $b$ e $f'(a),f''(a),\ldots$
    \end{itemize}
    \section{Formula di Taylor con resto di Lagrange}
        Dato $x_{0}\in I,\forall x\in I$ si ha:
        \begin{equation*}
            f(x) = \taylor{f}{x_{0}}{n}(x) + R_{n}f(x)
        \end{equation*}
        dove:
        \begin{equation*}
            R_{n}f(x) = f^{(n+1)}(c)(x-x_{0})^{n+1}\frac{1}{(n+1)!}
        \end{equation*}
        \subsection{Esercizi}
            \subsubsection{Esercizio 1}
                \textit{Calcolare un'approssimazione di $\sqrt{e}$ usando la formula di Taylor di ordine 5 per $f(x)=e^{x}$ e stimare l'errore.}
                \begin{equation*}
                    f(x) = e^{x} \quad \sqrt{e} = e^{\frac{1}{2}} = f\left(\frac{1}{2}\right)
                \end{equation*}
                Usiamo la formula di Taylor per stimare $f\left(\frac{1}{2}\right)$ prendendo $x=\frac{1}{2},n=5,x_{0}=0$
                \footnotesize
                \begin{equation*}
                    \begin{split}
                        f\left(\frac{1}{2}\right) &= \taylor{f}{0}{5}\left(\frac{1}{2}\right) + R_{5}f\left(\frac{1}{2}\right) \\
                        &= f(0)+f'(0)\frac{1}{2}+f''(0)\left(\frac{1}{2}\right)^{2}\frac{1}{2}+f'''(0)\left(\frac{1}{2}\right)^{3}\frac{1}{3!}+f^{(4)}(0)\left(\frac{1}{2}\right)^{4}\frac{1}{4!}+f^{(5)}(0)\left(\frac{1}{2}\right)^{5}\frac{1}{5!}+R_{5}f\left(\frac{1}{2}\right) \\
                        &= 1+\frac{1}{2}+\left(\frac{1}{2}\right)^{3}+\left(\frac{1}{2}\right)^{3}\frac{1}{6}+\left(\frac{1}{2}\right)^{4}\frac{1}{24}+\left(\frac{1}{2}\right)^{5}\frac{1}{120}+ e^{c}\left(\frac{1}{2}\right)^{6}\frac{1}{6!} \\
                        &= \frac{6331}{3840} + e^{c}\left(\frac{1}{2}\right)^{6}\frac{1}{6!} \\
                        &\approx 1.6486 + e^{c}\left(\frac{1}{2}\right)^{6}\frac{1}{6!}
                    \end{split}
                \end{equation*}
                \normalsize
                Quindi $\taylor{f}{0}{5}\left(\frac{1}{2}\right)\approx 1.6486$ fornisce un'approssimazione di $\sqrt{e}$.\\
                Per stimare l'errore, calcoliamo il massimo di $|R_{5}f(x)|$ sapendo che $c \in \left(0,\frac{1}{2}\right)$:
                \begin{equation*}
                    \begin{split}
                        \left|R_{5}f\left(\frac{1}{2}\right)\right| &= \left|e^{c}\left(\frac{1}{2}\right)^{6}\frac{1}{6!}\right| \leq e^{\frac{1}{2}}\left(\frac{1}{2}\right)^{6}\frac{1}{6!}
                    \end{split}
                \end{equation*}
                Ma $e < 3 \rightarrow \left|R_{5}f\left(\frac{1}{2}\right)\right|\leq\sqrt{3}\left(\frac{1}{2}\right)^{6}\frac{1}{6!}\xrightarrow{\sqrt{3} < 2}\left(\frac{1}{2}\right)^{5}\frac{1}{6!}\approx 0.00000434$
            \subsubsection{Esercizio 2}
                \textit{Calcolare un'approssimazione di $\ln\left(\frac{4}{3}\right)$ a  meno di  $2\cdot 10^{-3}$}
                \begin{equation*}
                    \begin{split}
                        &\ln\left(1+x\right)=f(x) \text{ è definita e derivabile infinite volte su } I=(-1,+\infty) \\
                        &\ln\left(\frac{4}{3}\right) = f\left(\frac{1}{3}\right)
                    \end{split}
                \end{equation*}
                Determinare $n\geq 1$ tale che $\left|f\left(\frac{1}{3}\right)-\taylor{f}{x_{0}}{n}\left(\frac{1}{3}\right)\right|<2\cdot 10^{-3}$, prendiamo $x_{0}=0$:
                \begin{equation*}
                    \taylor{f}{0}{n}(x) = f(0)+f'(0)x+f''(0)\frac{x^{2}}{2!}+\ldots+f^{(n)}(0)\frac{x^{n}}{n!}
                \end{equation*}
                Sappiamo che:
                \begin{equation*}
                    \begin{split}
                        &f'(x) = \frac{1}{1+x} \\
                        &f''(x) = -\frac{1}{(1+x)^{2}} \\
                        &f'''(x) = \frac{2}{(1+x)^{3}} \\
                        &f^{(4)}(x) = -\frac{6}{(1+x)^{4}} \\
                        &f^{(n)}(x) = (-1)^{n-1}\frac{(n-1)!}{(1+x)^{n}} \quad \forall n\geq 1 \Rightarrow f^{(n)}(0) = (-1)^{n-1}(n-1)!
                    \end{split}
                \end{equation*}
                Quindi:
                \begin{equation*}
                    \begin{split}
                        &\taylor{f}{0}{n}(x) = x-\frac{x^{2}}{2}+\frac{x^{3}}{3}-\ldots+(-1)^{n-1}\frac{x^{n}}{n} \\
                        &f^{(n)}(0)\frac{x^{n}}{n!}=(-1)^{n+1}(n-1)!\frac{x^{n}}{n!} = (-1)^{n+1}(n-1)!\frac{x^{n}}{n(n-1)!} = (-1)^{n+1}\frac{x^{n}}{n} \\
                        &\Rightarrow \taylor{f}{0}{n}(x)=x-\frac{x^{2}}{2}+\frac{x^{3}}{3}-\frac{x^{4}}{4}+\ldots+(-1)^{n-1}\frac{x^{n}}{n}
                    \end{split}
                \end{equation*}
                \begin{equation*}
                    \taylor{f}{0}{n}\left(\frac{1}{3}\right) = \frac{1}{3}-\left(\frac{1}{3}\right)^{2}\frac{1}{2}+\left(\frac{1}{3}\right)^{3}\frac{1}{3}-\left(\frac{1}{3}\right)^{4}\frac{1}{4}+\ldots+(-1)^{n-1}\left(\frac{1}{3}\right)^{n}\frac{1}{n}
                \end{equation*}
                Per trovare l'ordine $n$ tale che soddisfi la richiesta, calcoliamo il massimo di $|R_{n}f(x)|$:
                \begin{equation*}
                    \begin{split}
                        \left|R_{n}f\left(\frac{1}{3}\right)\right| &= \left|f^{(n+1)}(c)\left(\frac{1}{3}\right)^{n+1}\frac{1}{(n+1)!}\right| \\
                        &= \left|(-1)^{n+2}\frac{1}{(1+c)^{n+1}}\left(\frac{1}{3}\right)^{n+1}\frac{1}{n+1}\right| \\
                        &= \frac{1}{(1+c)^{n+1}}\left(\frac{1}{3}\right)^{n+1}\frac{1}{n+1} \leq \left(\frac{1}{3}\right)^{n+1}\frac{1}{n+1}\doteq \Delta_{n}
                    \end{split}
                \end{equation*}
                Calcoliamo $n\geq 1|\Delta_{n}\leq 2\cdot 10^{-3} = \frac{1}{500}$
                \begin{equation*}
                    \begin{split}
                        &n=1 \quad \Delta_{1} = \left(\frac{1}{3}\right)^{2}\frac{1}{2} = \frac{1}{18} \\
                        &n=2 \quad \Delta_{2} = \frac{1}{81} \\
                        &n=3 \quad \Delta_{3} = \frac{1}{324} \\
                        &n=4 \quad \Delta_{4} = \frac{1}{1215} \leq \frac{1}{500}\\
                    \end{split}
                \end{equation*}
                Quindi ora possiamo calcolare $\taylor{f}{0}{4}\left(\frac{1}{3}\right)$:
                \begin{equation*}
                    \taylor{f}{0}{4}\left(\frac{1}{3}\right) = \frac{1}{3}-\left(\frac{1}{3}\right)^{2}\frac{1}{2}+\left(\frac{1}{3}\right)^{3}\frac{1}{3}-\left(\frac{1}{3}\right)^{4}\frac{1}{4}
                \end{equation*}
    \section{Formula di Taylor con resto di Peano}
\end{document}