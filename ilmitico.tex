\documentclass[12pt, a4paper]{article}
\usepackage[utf8]{inputenc}
\newcommand\preamble{
    \usepackage[italian]{babel}
    \usepackage{geometry}
    \usepackage{amsmath}
    \usepackage{amssymb}
    \usepackage{graphicx}
    \usepackage{ulem}
    \geometry{margin=2cm}
    \usepackage{listings}
    \usepackage{xparse}
    \usepackage{expl3}
    \usepackage{tikz}
    \usepackage[colorlinks=true, linkcolor=blue]{hyperref}
    \usetikzlibrary{calc}
    \let\olditemize\itemize
    \renewcommand\itemize{\olditemize\setlength\itemsep{0em}}
    \geometry{a4paper, left=1.5cm, right=1.5cm, top=1cm, bottom=2cm}
}
\newcommand{\tikzmark}[1]{\tikz[baseline,remember picture] \coordinate (#1) {};}
\newcommand{\customfbox}[1]{
    \begin{center}
        \noindent\fbox{\parbox{\dimexpr\linewidth-2\fboxsep-2\fboxrule\relax}{\centering #1}}
    \end{center}
    }
\newcommand{\taylor}[3]{
    T_{#2,#3}^{#1}
}
\preamble

\begin{document}
\section{Formula di Taylor}
    \customfbox{$f^{(n)}$ indica l'ordine di derivazione $n$-esimo}
    Data una funzione $f:I\rightarrow\mathbb{R}$ definita su un intervallo $I$ e derivabile infinite volte ($C^{\infty}$)
    \subsection{Polinomio di Taylor di $f$ centrato in $x_0$ di ordine $n$}
        \begin{equation*}
            \taylor{f}{x_0}{n}(x) = f(x_0)+f'(x_0)(x-x_0)+f''(x_0)\frac{(x-x_0)^2}{2!}+\ldots+f^{(n)}(x_0)\frac{(x-x_0)^n}{n!}
        \end{equation*}
        % Eventuale esercizio qui
    \subsection{Formula di Taylor con resto di Lagrange}
        \begin{equation*}
            \begin{split}
                &f(x) = \taylor{f}{x_0}{n}(x) + R_nf(x)\\
                &R_nf(x) = \frac{f^{(n+1)}(c_n)(x-x_0)^{n+1}}{(n+1)!} \text{ con } c_n\in\{x_0,x\}
            \end{split}
        \end{equation*}
        Se $x=b$ e $x_0$ è vicino a $b$, $\taylor{f}{x_0}{n}(b)$ approssima $f(b)$ e $\left|f(b)-\taylor{f}{x_0}{n}(b)\right| = \left|R_nf(b)\right|$
        % Eventuale esercizio qui
    \subsection{Formula di Taylor con resti di Peano}
        \begin{equation*}
                f(x)= \taylor{f}{x_0}{n}(x) + (x-x_0)^n + \varepsilon(x-x_0) \text{ con } \lim_{x\rightarrow x_0} \varepsilon(x-x_0) = 0
        \end{equation*}
        % Eventuale esercizio qui
\section{Determinare se una serie converge/diverge o è indeterminata}
    \subsection{Serie armonica}
        \begin{equation*}
            \sum_{n=1}^{\infty}\frac{1}{n^\alpha} = \begin{cases}
                \text{converge} & \text{se }\alpha > 1\\
                +\infty & \text{se }\alpha\leq 1
            \end{cases}
        \end{equation*}
    \subsection{Serie geometrica}
        Dato $q\in\mathbb{R}$, la serie geometrica di ragione $q$ 
        \begin{equation*}
            \sum_{n=1}^{\infty}q^n = \begin{cases}
                +\infty & \text{se } q\geq 1\\
                \frac{q}{1-q} & \text{se } |q|< 1\\
                \text{indeterminata} & \text{se }q\leq -1
            \end{cases}
        \end{equation*}
        \subsection{Test dell'integrale}
            Sia $ f: [1, +\infty) \to \mathbb{R} $ una funzione:
            \begin{itemize}
                \item positiva
                \item decrescente
            \end{itemize}

            Posto $a_n = f(n)$ per ogni $n \geq 1$, allora:

            \begin{equation*}
                \sum_{n=1}^{\infty} a_n \text{ converge } \iff \int_{1}^{+\infty} f(x)dx \text{ converge}
            \end{equation*}

            Inoltre, valgono le disuguaglianze:
            \begin{equation*}
                \int_{1}^{+\infty} f(x)dx \leq \sum_{n=1}^{\infty} a_n \leq a_1 + \int_{1}^{+\infty} f(x)dx
            \end{equation*}
    \subsection{Condizione necessaria di Cauchy per le serie}
        \begin{equation*}
            \lim_{n\rightarrow+\infty}a_n = \begin{cases}
                0 & \Rightarrow \text{la serie \textbf{può} convergere o divergere} \\
                \neq 0 \text{ o non esiste} & \Rightarrow \text{la serie \textbf{diverge} sicuramente}
            \end{cases}
        \end{equation*}
        \textbf{Nota: } Questa condizione è \textbf{necessaria ma non sufficiente} alla convergenza della serie.
    \subsection{Criteri di convergenza/divergenza}
        \textit{Per serie a termini definitivamente positivi} ($\exists n | a_n\geq 0 \text{ con }n\in[n,+\infty)$)
        \begin{itemize}
            \item \textbf{Criterio del rapporto}: \begin{equation*}
                \lim_{n\rightarrow +\infty}\frac{a_{n+1}}{a_n} = l \in [0,+\infty)\cup\{+\infty\}
            \end{equation*} \begin{enumerate}
                \item $l<1\Rightarrow$ la serie converge
                \item $l>1\Rightarrow$ la serie diverge
            \end{enumerate}
            \item \textbf{Criterio della radice}: \begin{equation*}
                \lim_{n\rightarrow+\infty}\sqrt[n]{a_n} = l \in [0,+\infty)\cup\{+\infty\}
            \end{equation*} \begin{enumerate}
                \item $l<1\Rightarrow$ la serie converge
                \item $l>1\Rightarrow$ la serie diverge
            \end{enumerate}
            \item \textbf{Criterio del confronto}: supponiamo che $0\leq a_n\leq b_n$ allora \begin{equation*}
                \begin{split}
                    &\sum b_n \text{ converge} \Rightarrow \sum a_n \text{ converge}\\
                    &\sum a_n \text{ diverge} \Rightarrow \sum b_n \text{ diverge}
                \end{split}
            \end{equation*}
            \item \textbf{Criterio del confronto asintotico}: $a_n=c_nb_n$ dove $b_n,c_n\geq 0$ \begin{equation*}
                \begin{split}
                    &\lim_{n\rightarrow +\infty}c_n \in (0,+\infty) \Rightarrow \sum_{n=1}^{\infty}a_n \text{ e } \sum_{n=1}^{\infty}b_n \text{ hanno lo stesso carattere}\\
                    &\lim_{n\rightarrow +\infty}c_n = 0 \text{ e } \sum_{n=1}^{\infty}b_n \text{ converge } \Rightarrow \sum_{n=1}^{\infty}a_n \text{ converge}\\
                    &\lim_{n\rightarrow +\infty}c_n = l\in[0,+\infty)\cup\{+\infty\} \text{ e } \sum_{n=1}^{\infty}b_n \text{ diverge } \Rightarrow \sum_{n=1}^{\infty}a_n \text{ diverge}
                \end{split}
            \end{equation*}
        \end{itemize}
        \textit{Per serie a segno alterno}
        \begin{itemize}
            \item \textbf{Criterio di Leibnitz}: sia $\sum_{n=1}^{\infty}(-1)^{n+1}a_n = a_1-a_2+a_3-\ldots$ una serie tale che
                \begin{enumerate}
                    \item $a_n\geq 0\forall n\geq 1$
                    \item $\lim_{n\rightarrow+\infty}a_n=0$
                    \item $a_{n+1}\geq a_n \forall n\geq 1$
                \end{enumerate}
                allora converge e la somma $s$ della serie soddisfa $\left|s-s_n\right|\leq a_{n+1}\forall n\geq 1$.
        \end{itemize}
    \subsection{Convergenza assoluta}
        \begin{equation*}
            \sum_{n=1}^{\infty}a_n \text{ converge assolutamente se } \sum_{n=1}^{\infty}\left|a_n\right| \text{ converge}
        \end{equation*}
        Se la serie converge assolutamente, allora converge (semplicemente).
\section{Serie di Funzioni}
    Data la serie di funzioni:
    \begin{equation*}
        f(x) = \sum_{n=1}^{\infty}f_n(x)
    \end{equation*}
    \begin{enumerate}
        \item \textbf{Converge puntualmente} su $I$ se la serie converge $\forall x\in I$ e $f:I\rightarrow\mathbb{R}$
        \item \textbf{Converge assolutamente} su $I$ se la serie a termini positivi $\left(\left|f_n(x)\right|\right)$ converge $\forall x\in I$
    \end{enumerate}
    \textit{Se per qualche $x\in I$ la serie diverge o è indeterminata, bisogna restringere il dominio della funzione somma $f(x)$ all'insieme:} \begin{equation*}
        \left\{x\in I \mid \sum_{n=1}^{\infty}f_n(x) \text{ converge}\right\}
    \end{equation*}
    \subsection{Criterio di Weierstrass}
        Data una serie di funzioni, se la serie numerica a termini positivi converge:
        \begin{equation*}
            \sum_{n=1}^{\infty}\sup_{x\in I}\left|f_n(x)\right|
        \end{equation*}
        allora la serie di funzioni converge assolutamente e puntualmente (= totalmente) su $I$.
    \subsection{Corollari}
        \subsubsection{Continuità della somma}
            Se:
            \begin{enumerate}
                \item Le funzioni $f_n(x)$ sono continue su $I\forall n\geq 1$
                \item La serie converge totalmente su $I$ ad $f(x)$  
            \end{enumerate}
            Allora $f(x)$ è continua su $I$.
        \subsubsection{Integrazione temine a termine}
            Data una serie di funzioni dove le funzioni $f_n(x)$ sono definite su $[a,b]$ se:
            \begin{enumerate}
                \item Le funzioni $f_n$ sono continue su $[a,b]\forall n\geq 1$
                \item La serie converge totalmente su $[a,b]$ ad $f(x)$
            \end{enumerate}
            Allora $f(x)$ è integrabile in $[a,b]$:
            \begin{equation*}
                \int_{a}^{b}f(x)dx = \sum_{n=1}^{\infty}\int_{a}^{b}f_n(x)dx
            \end{equation*}
        \subsubsection{Derivazione termine a termine}
            Data una serie di funzioni dove le funzioni $f_n(x)$ sono definite su $I$ se:
            \begin{enumerate}
                \item Le funzioni $f_n$ sono derivabili e le derivate sono funzioni continue $\forall n\geq 1$
                \item La serie $\sum_{n=1}^{\infty}f_n(x)$ converge puntualmente su $I$
                \item La serie $\sum_{n=1}^{\infty}f'_n(x)$ converge totalmente su $I$
            \end{enumerate}
            Allora $f(x)$ derivabile e:
            \begin{equation*}
                f'(x) = \sum_{n=1}^{\infty}f'_n(x) \quad \forall x\in I
            \end{equation*}
        \subsubsection{Corollario 1.34}
            Data una serie di funzioni definite su $[a,b]$ se:
            \begin{enumerate}
                \item Le funzioni $f_n(x)$ sono continue su $[a,b]\forall n\geq 1$
                \item La serie converge totalamente su $(a,b)$
            \end{enumerate}
            Allora la serie converge totalmente su su $[a,b]$ e la funzione somma è continua su $[a,b]$
\section{Serie di potenze}
    La serie di funzioni
    \begin{equation*}
        \sum_{n=0}^{\infty}a_n x^n
    \end{equation*}            
    si chiama \textbf{serie di potenze di centro 0}
    \subsection{Raggio di convergenza}
        Data una serie di potenze $\sum_{n=0}^{\infty}a_n x^n \, \exists p \in [0,+\infty)\cup\{+\infty\}$, detto raggio di convergenza, tale che:
        \begin{itemize}
            \item se $p=0 \Rightarrow$ la serie di potenze converge solo in 0
            \item se $0<p<+\infty \Rightarrow$ la serie converge assolutamente su $(-p,p)$, non converge puntualmente su $(-\infty,p)\cup(p,+\infty)$, converge totalmente su $[-R,R] \,\forall 0<R<p$
            \item se $p=+\infty \Rightarrow$ la serie di potenze converge assolutamente su $\mathbb{R}$ e converge totalmente\\
            su $[-R,R] \, \forall R>0$
        \end{itemize}
        Se la serie converge puntualmente in $y\in\mathbb{R}, y\neq 0 \Rightarrow$ la serie converge totalmente su $[-R,R]\, \forall 0<R<\left|y\right|$.
        \subsubsection{Caso: Criteri del rapporto e della radice}
            Se esiste $l$:
            \begin{equation*}
                p=\begin{cases}
                    +\infty & l = 0\\
                    \frac{1}{l} & l\in(0,+\infty)\\
                    0  & l = +\infty
                \end{cases}
            \end{equation*}
    \subsection{Proprietà}
    \begin{itemize}
        \item La funzione somma \begin{equation*}
            f:I\rightarrow \mathbb{R} \qquad f(x)=\sum_{n=0}^{\infty}a_nx^n
        \end{equation*}
        è continua su $I$
        \item La funzione $f$ ammette primitiva sull'intervallo $I$ data da:
        \begin{equation*}
            \int f(x)dx = \int\left(\sum_{n=0}^{\infty}a_nx^n\right)dx=\sum_{n=0}^{\infty}a_n\frac{1}{n+1}x^{n+1}+c
        \end{equation*}
        dove la serie integrata ha raggio di convergenza $p$
        \item La funzione $f$ è derivabile su $(-p,p)$ e vale \begin{equation*}
            f'(x) = \left(\sum_{n=0}^{\infty}a_nx^n\right)'=\sum_{n=1}^{\infty}a_nnx^{n-1} \qquad x\in (-p,p)
        \end{equation*}
        dove la serie derivata ha raggio di convergenza $p$
        \item Data una serie di potenze con  raggio di convergenza $p>0$ la funzione somma $f(x)$ è derivabile infinite volte in $(-p,p)\forall k\in\mathbb{N}k\geq 0$: \begin{equation*}
            f^{(k)}(x)=\sum_{n=k}^{\infty}a_n\frac{n!}{(n-k)!}x^{n-k} \qquad x\in(-p,p)
        \end{equation*}
        In particolare: $f^{(n)}(0) = n!a_n \qquad n\in\mathbb{N}$
        \item Abbiamo due serie di potenze di centro $x_0$ e raggi di convergenza $p_1>0$ e $p_2>0$, rispettivamente. Se esiste $0<\delta<\min\{p_1,p_2\}\mid\forall x\in(-\delta,\delta)$:
        \begin{equation*}
            \sum_{n=0}^{\infty}a_nx^n = \sum_{n=0}^{\infty}b_nx^n 
        \end{equation*}
        allora $a_n=b_n\forall n\in\mathbb{N}$
        \item Sia $f: I\rightarrow\mathbb{R}$ con $I\subset\mathbb{R}$ tale che $f$ sia derivabile infinite volte in $I$.\\
        Se esistono $M>0,L>0\mid\left|f^{(n)(x)}\right|\leq M\cdot L^n \,\forall n\in\mathbb{N},\forall x\in I$ allora, fissato $x_0\in I$:
        \begin{equation*}
            \sum_{n=0}^{\infty}\frac{f^{(n)}(x_0)}{n!}(x-x_0)^n
        \end{equation*}
        si chiama serie di Taylor di $f$ con centro $x_0$ (se $x_0=0$ si chiama serie di Maclaurin). La funzione $f$ è sviluppabile in serie di Taylor con centro $x_0$ nell'intervallo $I$ se: \begin{equation*}
            f(x) = \sum_{n=0}^{\infty}\frac{f^{(n)}(x_0)}{n!}(x-x_0)^n \qquad \forall x\in I
        \end{equation*}
    \end{itemize}
\end{document}